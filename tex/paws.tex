\documentclass[11pt, a5paper]{article}
\usepackage[utf8]{inputenc}
\usepackage[spanish]{babel}
\usepackage[colorlinks=true,citecolor=black,linkcolor=black,urlcolor=black]{hyperref}
\usepackage[nottoc,numbib]{tocbibind}  % Para que aparezca la bibliografía en el índice
\usepackage{microtype}  % Para protrusión
\usepackage{xspace}
\usepackage[left=1.5cm,right=1.5cm,top=1.5cm,bottom=2cm]{geometry}

\newcommand{\quill}{\textsf{The Quill}\xspace}
\newcommand{\paw}{\textsf{The PAWS}\xspace}
\newcommand{\swan}{\textsf{SWAN}\xspace}
\newcommand{\daad}{\textsf{DAAD}\xspace}

\newcommand{\sistema}[1]{\noindent Sistema: #1 \nopagebreak}
\newcommand{\sistemas}[1]{\noindent Sistemas: #1 \nopagebreak}

\renewcommand{\baselinestretch}{0.85}  % Menor espacio interlineal
%\setcounter{secnumdepth}{1}  % No numerar las subsecciones
\setlength{\parskip}{1ex}  % Mayor espacio entre párrafos


\title{Guía unificada de los sistemas de creación de aventuras conversacionales
       \\\ \\\Huge{\quill, \paw, \swan y \daad}}
\author{\copyright\ 2010, 2018-2024 José Manuel Ferrer Ortiz
       \\\ \copyright\ 2023 José Luis Cebrián}

\begin{document}

\maketitle
\tableofcontents
\newpage

\begin{abstract}
El propósito de esta guía es servir como referencia del funcionamiento a nivel técnico de los sistemas de creación de aventuras conversacionales \quill, \paw, \swan y \daad, mostrando diferencias entre los diferentes sistemas, versiones y plataformas de los mismos, y sobre todo documentar características previamente no documentadas de ellos.
\end{abstract}

\section{Vocabulario}\label{vocab}

La longitud máxima de las palabras de vocabulario es de 4 letras en \quill, y de 5 letras en los demás sistemas. Esto supone que cualquier palabra introducida en una orden desde el teclado, será reducida por el intérprete a sus 4 ó 5 primeras letras, para su comparación con las palabras del vocabulario de la aventura.

El sistema \paw, introdujo la distinción de las palabras de vocabulario por tipo, permitiendo (y obligando) que el programador indique el tipo de cada palabra que añada al vocabulario, eligiendo uno de entre los siguientes 7 tipos: verbo, adverbio, nombre (sustantivo), adjetivo, preposición, conjunción y pronombre.

Esta característica fue heredada por los sucesores de \paw: \swan y \daad. Mientras que \quill, no hace distinción entre las palabras de vocabulario por su tipo (podríamos decir, que \quill sólo soporta un único tipo de palabras de vocabulario).

Cada palabra de vocabulario solamente debería aparecer una vez en la tabla de vocabulario\footnote{\ En realidad, Cozumel tiene su primera palabra de vocabulario repetida: \texttt{abajo}, incluso en las plataformas de 8 bits.}, para impedir que se le pueda asignar más de un tipo de palabra de vocabulario a la misma palabra. Pero se pueden definir nombres que puedan actuar también como verbos, cuando la orden no incluye ningún verbo, dándoles un número de palabra menor que 20 \cite[sección VOC]{PawsPC} (menor que 40 en DAAD versión 2).

En \paw y \daad los nombres por debajo de 50 se consideran nombres propios, con lo que el jugador no se puede referir a ellos con un pronombre.

A parte de los pronombres definidos como palabra de vocabulario, los intérpretes españoles reconocen también los pronombres enclíticos, buscando las terminaciones \texttt{-LA}, \texttt{-LO}, \texttt{-LAS}, \texttt{-LOS} cuando se usen sobre un verbo.


\section{Banderas}

Las banderas son posiciones de memoria en las que se almacena, en cada una de ellas, un valor numérico entero entre 0 y 255 (ambos incluidos). Estos valores se pueden consultar y modificar, y su función es dotar de vida a las aventuras, al mantener la mayor parte del estado de ejecución de las mismas.

\quill proporciona 33 de estas banderas (numeradas de 0 a 32), salvo en la plataforma Sinclair QL, donde proporciona 64 banderas (de la 0 a la 63 \cite{QuillQL}). Mientras que sus sucesores (\paw, \swan y \daad) proporcionan 256 banderas (que van desde la 0 hasta la 255).

Sin embargo, no todas las banderas están disponibles para cualquier propósito, ya que algunas de ellas tienen un significado predeterminado, y nos permitirán intercambiar con el intérprete información importante para la aventura.

Al inicio de la ejecución de la aventura (y cada vez que ésta se reinicia), el intérprete pone todas las banderas a su valor inicial, salvo en \daad versión 1 (al menos hasta la versión EGA de Jabato), donde el intérprete sólo inicializa las primeras 248 banderas (de la 0 a la 247), y deja las demás (de la 248 a la 255) sin cambiar.

El valor inicial es cero para todas las banderas, salvo la \nameref{flag1} (que sólo será cero si el jugador comienza la aventura con un inventario vacío) y, en los sistemas posteriores a \quill, también algunas otras de las banderas que tienen una utilidad predefinida; que podrán tomar valores iniciales distintos de cero.

A continuación, detallaremos el propósito que tiene cada bandera, de qué modo se utilizan y cómo afectarán a la aventura. Además, indicaremos su valor inicial, en caso de ser distinto de cero; y los sistemas (\quill, \paw, \swan o \daad) a los que se refieren las explicaciones.

\subsection{Bandera 0}

\sistemas{\quill, \paw, \swan y \daad}

Indica al intérprete si el jugador está en un sitio iluminado o, por el contrario, está en un lugar a oscuras. Un valor de cero significa luz, y cualquier valor distinto de cero significa oscuridad.

\sistemas{\quill, \paw, \swan y \daad versión 1}

En el momento de describir la localidad, si esta bandera tiene un valor distinto de cero (oscuridad), se decrementará la bandera 3 en una unidad. Si, además, no está presente el objeto 0 (que el intérprete considera como una fuente de luz), entonces se decrementará también en una unidad la bandera 4 y, en lugar de mostrarse la descripción de la localidad, se imprimirá el mensaje de sistema 0, que avisa al jugador de que se encuentra a oscuras.

Cada vez que transcurre un turno de juego, el intérprete comprueba si esta bandera indica oscuridad. Si es así, entonces decrementará en una unidad la bandera 9; y, si además, el objeto 0 está ausente, entonces decrementará también en una unidad la bandera 10.

\subsection{Bandera 1}\label{flag1}

\sistemas{\quill, \paw, \swan y \daad}

Contiene el número de objetos que el jugador lleva en su inventario. El intérprete inicializa esta bandera al valor correcto (el número de objetos que el jugador lleva al inicio de la aventura), y lo modificará automáticamente cada vez que el jugador coja o deje objetos.

\subsection{Bandera 2}

\sistemas{\quill, \paw, \swan y \daad versión 1}

Esta bandera es decrementada en una unidad por el intérprete cada vez que describe alguna localidad.

\subsection{Bandera 3}

\sistemas{\quill, \paw, \swan y \daad versión 1}

Esta bandera es decrementada en una unidad por el intérprete cada vez que describe alguna localidad en un sitio no iluminado (cuando la bandera 0 está a un valor distinto de cero).

\subsection{Bandera 4}

\sistemas{\quill, \paw, \swan y \daad versión 1}

Esta bandera es decrementada en una unidad por el intérprete cada vez que describe alguna localidad en un sitio no iluminado (cuando la bandera 0 está a un valor distinto de cero), pero sólo si tampoco está presente el objeto 0 (fuente de luz).

\subsection{Banderas 5 a 8}

\sistemas{\quill, \paw, \swan y \daad versión 1}

Estas banderas son decrementadas en una unidad por el intérprete cada vez que intenta obtener una frase para ejecutar.

La versión española de la guía técnica de \paw dice en el paso 5 - COGER LA FRASE, que son de la 7 a la 8, pero es una errata. En la guía técnica en inglés \cite{PawsPC} está bien.

\subsection{Bandera 9}

\sistemas{\quill, \paw, \swan y \daad versión 1}

Esta bandera es decrementada en una unidad por el intérprete cada vez que intenta obtener una frase para ejecutar, cuando el jugador está en un sitio no iluminado (cuando la bandera 0 está a un valor distinto de cero).

\subsection{Bandera 10}

\sistemas{\quill, \paw, \swan y \daad versión 1}

Esta bandera es decrementada en una unidad por el intérprete cada vez que intenta obtener una frase para ejecutar, cuando el jugador está en un sitio no iluminado (cuando la bandera 0 está a un valor distinto de cero) y además el objeto 0 (fuente de luz) está ausente.

\subsection{Banderas 11 a 28}

\sistemas{\quill, \paw y \daad}

Estas banderas no tienen ningún propósito predefinido y, por tanto, pueden ser utilizadas libremente por el programador.

\subsection{Bandera 29}\label{flag29}

\sistema{\quill}

Esta bandera no tiene ningún propósito predefinido y, por tanto, puede ser utilizada libremente por el programador.
\\\ \\
\sistema{\paw}

El intérprete comprueba el valor de esta bandera para saber cuándo debe dibujar los gráficos en pantalla. FIXME: Más detalle.

Para más información, consultar la Guía Técnica de \paw \cite{PawsZX}.
\\\ \\
\sistema{\swan}

El intérprete comprueba el valor de esta bandera al menos para saber cuándo debe dibujar los gráficos de localidad en pantalla. En concreto, este es el significado de los bits de esta bandera:

\begin{description}
  \item[Bit 3] Si se está mostrando ahora mismo (valor 1) o no (valor 0) en la pantalla - en la parte del gráfico de localidad - el menú por iconos de opciones del juego.
  \item[Bit 5] Desconocido, hace algo a la hora de imprimir mensajes.
  \item[Bit 6] Si está activada (valor 1) o desactivada (valor 0) la impresión de gráficos de localidad, lo cual se puede cambiar con el menú de opciones del juego.
  \item[Bit 7] Desconocido, lo modifica el intérprete (al menos el de PCW) al leer entrada de texto por parte del jugador, y al esperar una tecla. Tal vez lo use sólo de forma interna.
\end{description}

\sistema{\daad}

El intérprete inicializa esta bandera a 128 (o 129 si hay ratón) si está ejecutándose con un modo gráfico de 320 píxeles de ancho, con lo cual se dispondrá de 53 columnas para gráficos (de 6 píxeles de anchura) o texto. Si, en cambio, el intérprete se ejecuta en modo texto; entonces inicializará esta bandera a 0, indicando con ello que se dispondrá de 80 columnas para texto.

El resto de bits (inicialmente 0) tiene diferentes significados en el intérprete:

\begin{description}
  \item[Bit 7] Indica la presencia de un modo gráfico (valor 1) o sólo texto (valor 0)
  \item[Bit 6] (Sólo versión 1) Si está activado, ejecuta un CLS al dibujar el gráfico de la pantalla actual
  \item[Bit 5] Si está activada (valor 1) el comando DISPLAY es desactivado (no mostrará gráficos en pantalla)
  \item[Bit 3] (Sólo versiones 8 bits) Si está activada (valor 1), el comando DISPLAY modifica el color del borde de pantalla
  \item[Bit 0] (Sólo versión 2 y MS-DOS) Activada por el intérprete si se detectó un ratón conectado.
\end{description}

\subsection{Bandera 30}

\sistema{\quill en QL}

Esta bandera no tiene ningún propósito predefinido y, por tanto, puede ser utilizada libremente por el programador.
\\\ \\
\sistemas{\quill en otras plataformas, \paw, \swan y \daad}

En esta bandera se almacena la puntuación del jugador.

\subsection{Banderas 31 y 32}

\sistema{\quill en QL}

Estas banderas no tienen ningún propósito predefinido y, por tanto, pueden ser utilizadas libremente por el programador.
\\\ \\
\sistemas{\quill en otras plataformas, \paw y \daad}

Estas dos banderas contienen el número de turnos que han transcurrido desde el inicio del juego.

La bandera 31 contiene la parte menos significativa (LSB) y la 32 contiene la más significativa (MSB). Es decir, que cuando hayan pasado 255 turnos, la bandera 31 tendrá un valor de 255 y la 32 un valor de 0; y cuando se pase a 256 turnos, la bandera 31 guardará un 0 y la 32 un 1. Dicho de otra forma aún, el número total de turnos será 256 multiplicado por el valor de la bandera 32, más el valor de la bandera 31.

\subsection{Bandera 33}

\sistema{\quill en QL}

Esta bandera no tiene ningún propósito predefinido y, por tanto, puede ser utilizada libremente por el programador.
\\\ \\
\sistemas{\paw, \swan y \daad}

Contiene el verbo de la sentencia lógica actual.

Si se modifica desde la tabla de respuestas (o desde cualquier subproceso llamado desde esta), las entradas siguientes tendrán que encajar con esta nueva sentencia lógica recién modificada.

\subsection{Bandera 34}

\sistema{\quill en QL}

Esta bandera no tiene ningún propósito predefinido y, por tanto, puede ser utilizada libremente por el programador.
\\\ \\
\sistemas{\paw, \swan y \daad}

Contiene el primer nombre de la sentencia lógica actual.

Si se modifica desde la tabla de respuestas (o desde cualquier subproceso llamado desde esta), las entradas siguientes tendrán que encajar con esta nueva sentencia lógica recién modificada.

\subsection{Bandera 35}

\sistema{\quill en QL}

Esta bandera no tiene ningún propósito predefinido y, por tanto, puede ser utilizada libremente por el programador.
\\\ \\
\sistemas{\paw, \swan y \daad}

Contiene el adjetivo del primer nombre de la sentencia lógica actual.

\subsection{Bandera 36}

\sistema{\quill en QL}

Esta bandera no tiene ningún propósito predefinido y, por tanto, puede ser utilizada libremente por el programador.
\\\ \\
\sistemas{\paw, \swan y \daad}

Contiene el adverbio de la sentencia lógica actual.

\subsection{Bandera 37}

\sistema{\quill en QL}

Esta bandera no tiene ningún propósito predefinido y, por tanto, puede ser utilizada libremente por el programador.
\\\ \\
\sistemas{\paw, \swan y \daad}

Almacena el número máximo de objetos que el jugador es capaz de llevar (en su inventario) a la vez. Su valor inicial es 4, y puede ser modificado con el condacto ABILITY.

\subsection{Bandera 38}

\sistema{\quill en QL}

Esta bandera no tiene ningún propósito predefinido y, por tanto, puede ser utilizada libremente por el programador.
\\\ \\
\sistemas{\paw, \swan y \daad}

En esta bandera se mantiene el número de la localidad actual del jugador.

\subsection{Bandera 39}

\sistema{\quill en QL}

Esta bandera no tiene ningún propósito predefinido y, por tanto, puede ser utilizada libremente por el programador.
\\\ \\
\sistemas{\paw y \daad}

En esta bandera, se indica el número de la primera línea en la que se podrá escribir texto. Se puede modificar a través del condacto LINE.

TODO: Comprobar qué ocurre usando LINE junto con el condacto MODE, ver la pág. 34 de la Guía Técnica.

\paw en Spectrum parece que le asigna un valor inicial de 24.

\daad en la versión 1 de \daad, contiene el modo gráfico actual. En la versión DOS, los valores generalmente serán 4 para CGA (320x200 a 4 colores), 7 para texto monocromo (80x25), o 13 para EGA (320x200 a 16 colores). En Atari ST, generalmente contendrá 0 para modo de 320x200 a 16 colores, o 1 para modo 640x200 a 4 colores.

En la versión 2 esta bandera no se utiliza.

\subsection{Bandera 40}

\sistema{\quill en QL}

Esta bandera no tiene ningún propósito predefinido y, por tanto, puede ser utilizada libremente por el programador.
\\\ \\
\sistemas{\paw y \daad}

Algo relacionado con MODE... ¿el modo de pantalla? FIXME
Afecta a la parte en que el intérprete describe la localidad.

Según el manual de \daad, esta bandera no se utiliza, por lo que en las últimas versiones de este sistema es así.

\subsection{Bandera 41}

\sistema{\quill en QL}

Esta bandera no tiene ningún propósito predefinido y, por tanto, puede ser utilizada libremente por el programador.
\\\ \\
\sistema{\paw}

Algo relacionado con PROTECT... FIXME
¿Valor inicial en \paw\ != 0? FIXME
\\\ \\
\sistema{\daad}

Si es múltiplo de 8 (según la Guía Técnica de \daad \cite[pág. 61]{DAAD}), las órdenes que introduzca el jugador quedarán intercaladas entre los textos de la aventura. Si no, las órdenes se obtendrán en la línea inferior de la ventana de impresión seleccionada, sin afectar al cursor de ésta, y la orden introducida se borrará de la pantalla al terminar de procesarla.

\subsection{Bandera 42}

\sistema{\quill en QL}

Esta bandera no tiene ningún propósito predefinido y, por tanto, puede ser utilizada libremente por el programador.
\\\ \\
\sistemas{\paw y \daad}

Esta bandera guarda el número del mensaje de sistema que se imprimirá para alentar o guiar al jugador a que introduzca una nueva orden. Se puede cambiar con el condacto PROMPT (inexistente en \daad versión 2).

El valor 0 (que es el valor por defecto) hace que el intérprete elija un mensaje aleatoriamente entre los mensajes de sistema 2, 3, 4 y 5; con una probabilidad de 3/10, 3/10, 3/10 y 1/10, respectivamente.

\subsection{Bandera 43}

\sistema{\quill en QL}

Esta bandera no tiene ningún propósito predefinido y, por tanto, puede ser utilizada libremente por el programador.
\\\ \\
\sistemas{\paw, \swan y \daad}

Contiene la preposición de la sentencia lógica actual.

\subsection{Bandera 44}

\sistema{\quill en QL}

Esta bandera no tiene ningún propósito predefinido y, por tanto, puede ser utilizada libremente por el programador.
\\\ \\
\sistemas{\paw, \swan y \daad}

Contiene el segundo nombre de la sentencia lógica actual.

\subsection{Bandera 45}

\sistema{\quill en QL}

Esta bandera no tiene ningún propósito predefinido y, por tanto, puede ser utilizada libremente por el programador.
\\\ \\
\sistemas{\paw, \swan y \daad}

Contiene el adjetivo del segundo nombre de la sentencia lógica actual.

FIXME: ¿el adjetivo del segundo nombre, o simplemente el segundo adjetivo (el que se escriba en segunda posición)?

\subsection{Banderas 46 y 47}

\sistema{\quill en QL}

Estas banderas no tienen ningún propósito predefinido y, por tanto, pueden ser utilizadas libremente por el programador.
\\\ \\
\sistemas{\paw y \daad}

Contienen el nombre y adjetivo (respectivamente) al que se puede referir el jugador con un pronombre (véase \nameref{vocab}).

\subsection{Banderas 48 y 49}

\sistema{\quill en QL}

Estas banderas no tienen ningún propósito predefinido y, por tanto, pueden ser utilizadas libremente por el programador.
\\\ \\
\sistemas{\paw y \daad}

Algo relacionado con el tiempo muerto (condacto TIMEOUT): FIXME

\subsection{Bandera 50}

\sistema{\quill en QL}

Esta bandera no tiene ningún propósito predefinido y, por tanto, puede ser utilizada libremente por el programador.
\\\ \\
\sistemas{\paw y \daad}

Algo relacionado con el condacto DOALL: FIXME

\subsection{Bandera 51}

\sistema{\quill en QL}

Esta bandera no tiene ningún propósito predefinido y, por tanto, puede ser utilizada libremente por el programador.
\\\ \\
\sistemas{\paw y \daad}

En esta bandera se guarda el código del último objeto referido, normalmente por parte del jugador, obtenido mediante WHATO, pero el programador lo puede cambiar por algún otro objeto.

Cuando esta bandera es rellenada por WHATO o SETCO, se rellenan a su vez las banderas 54 a 57 (y también 58/59 en \daad versión 2), con información sobre el objeto referido.

Es utilizado por algunos condactos, al menos por PUTO. Será imprimido el nombre de este objeto en lugar de \_ en los mensajes.

\subsection{Bandera 52}

\sistema{\quill en QL}

Esta bandera no tiene ningún propósito predefinido y, por tanto, puede ser utilizada libremente por el programador.
\\\ \\
\sistemas{\paw, \swan y \daad}

Almacena el peso máximo que el jugador es capaz de cargar en su inventario (contando solamente el peso los objetos que lleve ahí, no los que lleve puestos) a la vez. Su valor inicial es 10, y puede ser modificado con el condacto ABILITY.

\subsection{Bandera 53}

\sistema{\quill en QL}

Esta bandera no tiene ningún propósito predefinido y, por tanto, puede ser utilizada libremente por el programador.
\\\ \\
\sistemas{\paw y \daad}

Con esta bandera, se le indica al intérprete el modo en que deseamos que liste los objetos que están en el inventario del jugador, ¿cuando usemos el condacto INVEN?. FIXME: Más detalle. Afecta a los condactos LISTOBJ y LISTAT.

\subsection{Bandera 54}

\sistema{\quill en QL}

Esta bandera no tiene ningún propósito predefinido y, por tanto, puede ser utilizada libremente por el programador.
\\\ \\
\sistemas{\paw y \daad}

El valor de esta bandera es el código de la localidad en que se encuentra el objeto actualmente referido (el objeto guardado en la bandera 51).

\subsection{Bandera 55}

\sistema{\quill en QL}

Esta bandera no tiene ningún propósito predefinido y, por tanto, puede ser utilizada libremente por el programador.
\\\ \\
\sistemas{\paw y \daad}

Esta bandera proporciona el peso que tiene el objeto actualmente referido (el objeto guardado en la bandera 51).

\subsection{Bandera 56}

\sistema{\quill en QL}

Esta bandera no tiene ningún propósito predefinido y, por tanto, puede ser utilizada libremente por el programador.
\\\ \\
\sistemas{\paw y \daad}

Esta bandera contendrá 128 si el objeto actualmente referido (el guardado en la bandera 51) es un contenedor.

\subsection{Bandera 57}

\sistema{\quill en QL}

Esta bandera no tiene ningún propósito predefinido y, por tanto, puede ser utilizada libremente por el programador.
\\\ \\
\sistemas{\paw y \daad}

Esta bandera contendrá 128 si el objeto actualmente referido (el guardado en la bandera 51) es una prenda.

\subsection{Bandera 58}

\sistema{\quill en QL}

Esta bandera no tiene ningún propósito predefinido y, por tanto, puede ser utilizada libremente por el programador.
\\\ \\
\sistema{\paw}

Esta bandera, inicialmente documentada como reservada para futuras ampliaciones, fue utilizada en las versiones de \paw A16 y posteriores, como parte del soporte para dar órdenes a los PSI. \cite{PawsSupl}
\\\ \\
\sistema{\swan}

Esta bandera se utiliza para guardar el número de parte en ejecución. El intérprete no la cambia por sí mismo (salvo en la inicialización de banderas), pero sí que toma ese valor cuando se carga una partida, para cargar y cambiar a la parte que corresponda - es decir, a la parte cuyo número es el valor para la bandera 58 almacenado en la partida guardada.
\\\ \\
\sistema{\daad}

En aventuras con versión de formato de base de datos 2, a esta bandera se copia el primer byte de los atributos extra del objeto referido. Lo rellena WHATO.

\subsection{Bandera 59}

\sistema{\quill en QL}

Esta bandera no tiene ningún propósito predefinido y, por tanto, puede ser utilizada libremente por el programador.
\\\ \\
\sistema{\paw}

La Guía Técnica de \paw \cite{PawsZX} recomienda no utilizar esta bandera, puesto que está reservada para ampliaciones futuras.
\\\ \\
\sistema{\daad}

En aventuras con versión de formato de base de datos 2, a esta bandera se copia el segundo byte de los atributos extra del objeto referido. Lo rellena WHATO.

\subsection{Bandera 60}

\sistema{\quill en QL}

En esta bandera se almacena la puntuación del jugador.
\\\ \\
\sistema{\paw}

Esta bandera no tiene ningún propósito predefinido y, por tanto, puede ser utilizada libremente por el programador.
\\\ \\
\sistema{\daad}

El condacto INKEY guardará aquí el código de la tecla que el jugador estuviese pulsando. En plataformas de 16 bits, si la tecla pulsada es especial, se guarda su código en la bandera 61, y un 0 en esta bandera. \cite[págs. 25, 61 y 62]{DAAD}

\subsection{Bandera 61}

\sistema{\quill en QL}

Esta bandera contiene la parte menos significativa (LSB) del número de turnos que han transcurrido desde el inicio del juego.

Es decir, que cuando hayan pasado 255 turnos, esta bandera tendrá un valor de 255 y la 62 un valor de 0; y cuando se pase a 256 turnos, esta bandera guardará un 0 y la 62 un 1. Dicho de otra forma aún, el número total de turnos será 256 multiplicado por el valor de la bandera 62, más el valor de esta bandera.
\\\ \\
\sistema{\paw}

Esta bandera no tiene ningún propósito predefinido y, por tanto, puede ser utilizada libremente por el programador.
\\\ \\
\sistema{\daad}

En plataformas de 16 bits, si la tecla pulsada es especial, el condacto INKEY guardará aquí el código de la tecla que el jugador estuviese pulsando, y un 0 en la bandera 60. \cite[págs. 25, 61 y 62]{DAAD}

\subsection{Bandera 62}

\sistema{\quill en QL}

Esta bandera contiene la parte más significativa (MSB) del número de turnos que han transcurrido desde el inicio del juego.
\\\ \\
\sistema{\paw}

Esta bandera no tiene ningún propósito predefinido y, por tanto, puede ser utilizada libremente por el programador.
\\\ \\
\sistema{\daad}

En plataformas Atari ST y PC, aquí se refleja cuál es el modo gráfico actual de la máquina. \cite[pág. 62]{DAAD}

En Atari ST, 0 indica modo de baja resolución (320x200 a 16 colores) y 1, alta resolución (640x200 a 4 colores). En modo de alta resolución, \daad utiliza un juego de caracteres de 8x8 en lugar de 6x8 (y las posiciones de las ventanas se ajustan de forma acorde) y no soporta gráficos.

En PC, los valores posibles son 4 (CGA 320x200 a 4 colores), 7 (texto monocromo 80x25), 13 (EGA 320x200 a 16 colores) y 141 (VGA 320x200 a 16 colores, con paleta editable; 141 equivale a 13 + 128, el bit 7 se utiliza para indicar la presencia de paleta de colores modificable).

\subsection{Bandera 63}

\sistemas{\quill en QL y \paw}

Esta bandera no tiene ningún propósito predefinido y, por tanto, puede ser utilizada libremente por el programador.
\\\ \\
\sistema{\daad}

El intérprete guarda aquí el número de la ventana de impresión activa, al menos en las últimas versiones de \daad. \cite[pág. 62]{DAAD}

Este valor se actualiza por el condact WINDOW y su propósito es informativo. Es decir, modificarlo no tiene ningún efecto en el intérprete.

\subsection{Banderas 64 a 255}

\sistemas{\paw y \daad}

Estas banderas no tienen ningún propósito predefinido y, por tanto, pueden ser utilizadas libremente por el programador.

\subsection{Banderas 248 a 255}

\sistema{\daad versión 1}

Estas banderas conservan su valor incluso al reiniciar la aventura.


\section{Mensajes de sistema}

Los mensajes de sistema son mensajes de información que se le ofrecen al jugador, en distintos contextos. A diferencia de los mensajes de usuario, algunos de estos mensajes son mostrados automáticamente por el intérprete, aunque las aventuras de \paw y \daad pueden mostrarlos también en cualquier momento, haciendo uso de la acción SYSMESS (no existente en \quill).

\quill tiene 32 mensajes de sistema, todos ellos con funcionalidades determinadas de antemano. Pero \paw permite hasta 256 mensajes de sistema, de los cuales menos de una cuarta parte tiene finalidades predeterminadas, pudiendo los demás utilizarse libremente. Esto, junto con los mensajes de usuario, deja al programador de aventuras \paw con más de 440 mensajes de propósito general, mientras que \quill sólo proporciona los 256 mensajes de usuario.

Con lo que el autor de este documento sabe de momento de \daad, este sistema proporciona al menos el mismo número total de mensajes de propósito general (todos los de usuario y los libres de sistema). Aunque, Javier San José en un mensaje \cite{JSJ} de la lista de correo de CAAD, hablaba del soporte de ¡hasta 256 tablas de mensajes! (supongo, contando la de sistema):

\begin{quote}
\guillemotleft Además se ampliaba el número de tablas de mensajes. Recuerdo que en PAWS sólo había dos tablas de mensajes (la de usuario y la del sistema) con un máximo de 256 mensajes cada una. En \daad se podía crear hasta 256 tablas de mensajes con 256 mensajes cada una.\guillemotright
\end{quote}

\subsection{Mensaje 32}

\sistema{\daad}

Como descubrió Daniel Carbonell, este mensaje de sistema debe tener valor (no puede ser una cadena vacía), de lo contrario en textos largos, el intérprete puede omitir la última línea en pantalla justo antes de paginar. Al menos, en el caso en que esa última línea del bloque termine en la última columna de la ventana de impresión.

\subsection{Mensaje 33}

\sistemas{\paw y \daad}

A pesar de no verse afectado por el uso del condacto PROMPT, éste es el verdadero \emph{prompt} en el sentido original del término (consúltese \url{http://es.wikipedia.org/wiki/Prompt}). Como tal, será mostrado por el intérprete siempre que quede a la espera de recibir cualquier línea de entrada por parte del jugador, a través del teclado.


\section{Condactos}

Hay dos tipos condactos: condiciones y acciones. De hecho, la palabra condacto (en inglés, \emph{condact}) proviene de la unión de esas dos palabras en inglés: condition (condición) y action (acción).

\subsection{Condiciones}

Las condiciones son condactos que cambian el flujo de ejecución impidiendo que se ejecute el siguiente condacto en su bloque, dependiendo de si se cumple o no alguna condición determinada.

Las condiciones de cada sistema y sus finalidades son:

\subsubsection{CHANCE porcentaje}

\sistemas{\quill, \paw y \daad}

Siendo \emph{porcentaje} un valor entre 1 y 99, ambos incluidos, esta condición se cumple si este número es mayor o igual\footnote{\ Los manuales de \quill y \paw dicen aquí ``menor o igual'', pero se trata de una errata que ha pasado desapercibida.} que un número aleatorio entre 1 y 100, también inclusive.

Es decir, que una condición \texttt{CHANCE 25} será satisfactoria un 25\% de las veces, y saltará en las demás ocasiones (el 75\% de veces restante).


\subsection{Acciones}

Las acciones son condactos que normalmente no cambian el flujo de ejecución, y que afectan a la variable del intérprete que indica que se hizo algo (la que se consulta con las condiciones ISDONE e ISNDONE).

Salvo que se indique lo contrario en esta sección (o no sea conocido todavía), las acciones que están aquí no cambian el flujo de ejecución, y dan valor verdadero a la variable que indica que se hizo algo.

Las acciones de cada sistema y sus finalidades son:

\subsubsection{ANYKEY}

\sistema{\quill}

Imprime el mensaje de sistema 16 en las dos últimas líneas de la pantalla, y espera a que se pulse una tecla. ¿Luego borra el mensaje?

\sistema{\paw}

Imprime el mensaje de sistema 16 en las dos últimas líneas de la pantalla, espera a que se pulse una tecla o se supere el tiempo muerto, y borra el mensaje.

\sistemas{\swan y \daad}

Imprime el mensaje de sistema 16, y espera a que se pulse una tecla o se supere el tiempo muerto. Al menos en Jabato, luego no borra el mensaje.

Al menos en \daad, este condacto no tiene en cuenta pulsaciones de teclas modificadoras, al menos no tiene en cuenta: Ctrl, Shift, Alt, Teclas de Windows de abajo (comprobado bajo DOSBox, habría que ver desde Windows).

\subsubsection{CLS}

En \daad, esta acción limpia un recuadro del contenido de la pantalla, en concreto el de la ventana de impresión seleccionada.

Al menos bajo DOS, por hacer uso de la interrupción Xh Y (TODO: anotar cuál es), limpiará un recuadro de múltiplos de $8 \times 8$ píxeles. Por esto, al ser las columnas en las ventanas de impresión de 6 píxeles de ancho, implica que se limpiará en ocasiones más que solamente la ventana de impresión, limpiando hasta 6 píxeles de ancho de más. El ancho del recuadro a borrar, es el mínimo múltiplo de 8 píxeles que sea superior o igual al ancho en píxeles de la ventana de impresión.

\subsubsection{INVEN}

\sistemas{\quill y \paw}

Inexistente en \daad.

\subsubsection{PROCESS num\_proceso}

\sistemas{\paw y \daad}

Comprobado con \daad (¿versión 1?), desde la tabla de proceso 1 no se permite llamar ni al proceso 0 ni al 2, ni a un número de proceso inexistente (en todos estos casos da error 6).

\sistema{\daad}

Esta acción quita la marca de que se hizo algo (en la variable del intérprete para este propósito), lo cual permitirá comprobar su valor (con las condiciones ISDONE e ISNDONE) cuando el flujo de ejecución regrese a la tabla desde donde se ejecutó PROCESS.

\subsubsection{SKIP salto}

\sistema{\daad versión 2}

Cambia el flujo de ejecución saltando al inicio de la entrada de proceso (de la actual tabla de proceso) según el valor del parámetro, que indica el índice de la tabla a la que saltar en relación a la siguiente entrada a la actual (la entrada donde está la acción SKIP que se está ejecutando).

Es decir, el valor que poner al parámetro debe ser la diferencia entre el índice (número de entrada en el proceso) de la entrada siguiente a la actual, y el de la entrada a la que se quiere saltar.

De esta manera, un valor para \texttt{salto} 0 indica saltar al inicio de la siguiente entrada, un valor de -1 indica saltar al inicio de la entrada actual, -2 es saltar a la entrada anterior, y 1 es saltar a la entrada posterior a la siguiente de la actual (evitando así la ejecución del resto de la entrada actual, y la totalidad de la entrada siguiente).

El parámetro de SKIP admite valores entre -128 y 127.

\subsubsection{SWAP num\_objeto num\_objeto}

\sistemas{\quill, \paw, \swan y \daad}

Intercambia las localidades donde se encuentran los objetos dados como parámetro.

En \daad, el compilador DC rescatado no soporta indirección para esta acción, mientras que los intérpretes lo soportan sin problema.

\subsubsection{TIME}

\sistema{\paw}

En la versión Spectrum de \paw, la duración es en intervalos de 1,28 segundos \cite{PawsZX}, mientras que en las demás versiones, es en intervalos de 1 segundo \cite{PawsPC}.

\subsubsection{WINDOW num\_ventana}

\sistema{\daad}

Selecciona la ventana de impresión de número \emph{num\_ventana} para trabajar sobre ella. Cualquier tipo de impresión tanto de texto como de gráficos, se realiza siempre sobre la ventana seleccionada.

\daad proporciona ocho ventanas de impresión, numeradas de la 0 a la 7, siendo la número 1 la seleccionada inicialmente. Para cada una de estas ventanas, guarda la siguiente información:

\begin{description}
  \item[Coordenadas de origen] Por defecto: [0, 0]. Se cambian a través del condacto WINPOS.

  \item[Tamaño de la ventana] Por defecto: [25, 53] en modo gráfico y [25, 80] en modo texto (véase la información de la \nameref{flag29}). Se cambia a través del condacto WINSIZE.

  \item[Coordenadas del cursor] Por defecto: [0, 0]. Se cambian a través del condacto PRINTAT.
\end{description}

Los condactos PRINTAT, WINAT y WINSIZE modifican solamente la ventana seleccionada.

FIXME: ¿Se puede redimensionar la ventana 0?


\section{Ejecución}

Esto es lo que realiza el intérprete al ejecutarse, en el siguiente orden secuencial, salvo que ocurran cambios de flujo debidos a acciones.

\subsection{Inicializar}

\sistemas{\quill, \paw, \swan y \daad}

\subsection{Describir localidad actual}

\sistemas{\quill, \paw, \swan y \daad versión 1}

\subsection{Iterar sobre tabla de estado}

\sistema{\quill}

\subsection{Iterar sobre tabla de procesos 1}

\sistemas{\paw, \swan y \daad versión 1}

Como se ejecuta justo después de describir la localidad actual, esta tabla de procesos es la que permite añadir textos circunstanciales, listar objetos presentes, etc.

\subsection{Iterar sobre tabla de procesos 2}

\sistemas{\paw, \swan y \daad versión 1}

Esta tabla de procesos se utiliza para código que se ejecuta cada turno.

\subsection{Obtener orden}

\sistemas{\quill, \paw, \swan y \daad versión 1}

\subsection{Iterar sobre tabla de procesos 0}

\sistemas{\paw, \swan y \daad}

Esta tabla de procesos es la tabla de respuestas en \paw, \swan y la versión 1 de \daad. En la versión 2 de \daad, esta tabla es el bucle principal de la aventura, el punto de inicio y fin del programa.

\subsection{Buscar en tabla de conexiones}

\sistemas{\quill, \paw y \swan}

\subsection{Iterar sobre tabla de eventos}

\sistema{\quill}


\begin{thebibliography}{99}

\bibitem{QuillZX}
  Graeme Yeandle,
  ``\emph{The Quill: An Adventure System for the 48K Spectrum}''.
  Gilsoft International Ltd.,
  Serial C,
  1983.\\
  \url{http://www.worldofspectrum.org/pub/sinclair/games-info/q/%
       QuillAdventureSystemThe.txt}

\bibitem{QuillCPC}
  Graeme Yeandle,
  ``\emph{The Quill: An Adventure System for the Amstrad CPC 464}''.
  Gilsoft International Ltd.,
  Serial A,
  1983.\\
  \url{http://www.cpcwiki.eu/imgs/c/cf/The_Quill_%28Gilsoft%29_Manual.pdf}\\
  \url{http://web.archive.org/web/www.cpcwiki.eu/imgs/c/cf/The_Quill_(Gilsoft)_Manual.pdf}

\bibitem{QuillApple}
  Graeme Yeandle,
  ``\emph{AdventureWriter\texttrademark for the Apple II, II+, IIe, IIc and Franklin Ace 1000 computers}''.
  Gilsoft International Ltd.,
  1983.\\
  \url{http://archive.org/details/vgmuseum_miscgame_advwriter-manual}

\bibitem{QuillBBC}
  ``\emph{The Quill Adventure Writing System for the BBC B \& Electron}''.
  Gilsoft International Ltd.,
  1984.\\
  \url{http://www.nvg.ntnu.no/bbc/sw/Various/Quill.doc}

\bibitem{QuillOric}
  Tim J. Gilberts \& Graeme Yeandle,
  ``\emph{The Quill: An Adventure Writing System by Graeme Yeandle. Oric 1 /
  ATMOS version by Tim Gilberts}''.
  Gilsoft International Ltd.,
  Serial A,
  1985.\\
  \url{https://mocagh.org/miscgame/quill-manual.pdf}

\bibitem{QuillQL}
  Huw H. Powell,
  ``\emph{The Quill: An Adventure System for the Sinclair QL by Huw H.Powell}''.
  Gilsoft International Ltd.,
  1986.\\
  \url{https://raw.githubusercontent.com/alvaroalea/QL_TheQuill_Tools/main/info/QL_Manual.txt}

\bibitem{QuillAWRef}
  Gareth Pitchford,
  ``\emph{The Quill (and AdventureWriter) Reference Guide}''.
  8bitag.com,
  Version 1.0,
  octubre de 2020.\\
  \url{http://8bitag.com/info/documents/Quill-AdventureWriter-Reference.pdf}

\bibitem{PawsZX}
  Tim J. Gilberts,
  ``\emph{A Technical guide to: The Professional Adventure Writer. A graphic
  adventure writing system for the Sinclair Spectrum computers}''.
  Gilsoft International Ltd.,
  1986.\\
  \url{http://worldofspectrum.org/pub/sinclair/games-info/p/%
       ProfessionalAdventureWriter_TechnicalGuide.zip}

\bibitem{PawsSupl}
  ``\emph{THE PROFESSIONAL ADVENTURE WRITING SYSTEM. A supplement for the
  Spectrum -- Version A 16}''.\\
  \url{http://worldofspectrum.org/pub/sinclair/games-info/p/%
       ProfessionalAdventureWriter_A16Supplement.txt}

\bibitem{PawsPC}
  Tim J. Gilberts \& Graeme Yeandle,
  ``\emph{PC Adventure Writer -- Technical Guide}''.\\
  \url{http://www.yeandle.plus.com/advent/pawtech.html}\\
  \url{http://web.archive.org/web/www.yeandle.plus.com/advent/pawtech.html}

\bibitem{JSJ}
  Javier San José,
  ``\emph{RE: el DAAD}''.
  Lista de correo del Club de Aventuras AD,
  29 de noviembre de 2002.\\
  \url{http://www.caad.es/listas/caad/msg12353.html}

\bibitem{DAAD}
  ``\emph{A Technical guide to: DAAD Adventure Writer - Version 2 Release 1. A multi-machine adventure writing system. Revised in November '91}''.
  Infinite Imaginations,
  1988-1991.

\end{thebibliography}

\end{document}